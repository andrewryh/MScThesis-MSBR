\chapter[Introduction]{Introduction}

\section{Background and motivation}

Nowadays humankind has only few ways to generate reliable, nonintermittent base load power, namely, in no particular order: fossil fuel, hydroelectric, biomass, geothermal, and nuclear energy. Because of increasing global warming and climate change concerns, sources that have negligible CO$_2$ footprints represent crucial measures for control global temperature change. From environmental viewpoint these observations allow conclude that hydro and nuclear power are preferable ways to generate nonintermittent power\footnote{Biomass could be classified as renewable energy source but recent research has proved that it also produce considerable amount of CO$_2$ and not necessarily beneficial for the environment over time scales that are relevant when considering global climate change \cite{cherubini_co2_2011}.}. Nevertheless, potential for hydro power is strictly limited by local geographical conditions and have significant impact on environment, affecting land use, homes, and natural habitats in the dam area, consequently, one could argue that nuclear power should find wide usage around the world. 
 
In reality, nuclear power plants currenly generate 4.9\% of the global energy production \cite{noauthor_key_2017}, a figure which is projected to stay constant up to 2040 while electricity demand predicted to expand by 30\% \cite{noauthor_world_2017}. In contrast, contribution of coal in 2040 projected to be about 30\%. This visible fear towards large-scale switch from fossil to nuclear power based on a number factors such as concerns regarding safety, nuclear weapon prolifiration, radioactive waste treatment, and competitiveness with other sources of energy (i.e. renewables). Unfortunately, because of all these fears, a negative public attitude to nuclear has formed in many developed countries which makes challenging to advocate its zero emissions benefits.

\glspl{MSR} is one effort to overcoming the problems related to conventional \glspl{NPP}. As one of six advanced reactor concepts that have been chosen by the Generation IV International Forum (GIF) for further research and development, \glspl{MSR}  offering significant improvements ``in the four broad areas of sustainability, economics, safety and reliability, and proliferation resistance and physical protection" \cite{doe_technology_2002}. To achieve the goals formulated by the GIF, \glspl{MSR} attempt to simplify the reactor core and improve ingerent safety by using liquid coolant which is also a fuel\footnote{Molten-salt-cooled reactors with solid fuel are usually also referred to \glspl{MSR}. However, in this thesis, \glspl{MSR} are assumed to be reactors with liquid fuel which simultaneously serves as coolant.}. In the thermal spectrum \gls{MSR}, fluorides of fissile and/or fertile materials (i.e. UF$_4$, PuF$_3$ and/or ThF$_4$) are mixed with carrier salts to form a liquid fuel which is circulated in a loop-type primary circuit \cite{haubenreich_experience_1970}. This innovation leads to immediate advantages over traditional, solid-fueled, reactors. These include near-atmospheric pressure in the primary loop, relatively high coolant temperature, outstanding neutron economy, a high level of inherent safety, reduced fuel preprocessing, and the ability to continuously remove fission products and add fissile and/or fertile elements \cite{leblanc_molten_2010}. 

The thermal spectrum \gls{MSBR} was designed to realize the promise of the thorium fuel cycle, which uses of natural thorium instead of enriched uranium. Thorium breeds fissile $^{233}$U and avoids uranium enrichment. The mixture of LiF-BeF$_2$-ThF$_4$-UF$_4$-PuF$_3$ has a melting point of $499^\circ$C, a low vapor pressure at operating temperatures, and good flow and heat transfer properties \cite{robertson_conceptual_1971}. 

The \gls{MSBR} complex geometry is challenging to describe in software input and usually require major geometric 
approximations \cite{park_whole_2015}. 

\section{Objectives}
This thesis introduces online reprocessing simulation code SaltProc which expands the capability of a continuous-energy Monte Carlo Burnup calculation code SERPENT 2 for simulating liquid-fueled \gls{MSR} operation \cite{andrei_rykhlevskii_arfc/saltproc:_2018}. The thesis also reports the application of the SaltProc-SERPENT 2 workbench to the \gls{MSBR}, which represents the continuation of the work presented in \cite{rykhlevskii_full-core_2017}, \cite{rykhlevskii_online_2017}. The main objective of the thesis herein is to analyse \gls{MSBR} neutronics and fuel cycle to find the equilibrium core composition and core depletion. The secondary objective is to compare predicted operational and safety parameters of the \gls{MSBR} at initial and equilibrium state. A tertiary goal is to demonstrate and prove that in one-fluid two-region \gls{MSBR} conceptual design the undermoderated outer core zone II works as virtual ``blanket" which main purpose is to reduce neutron leakage and improve breeding ratio due to neutron energy spectral shift. Finally, $^{232}$Th feed rate should be determined to estimate thorium-fueled \gls{MSBR} fuel cycle performance.

\section{Methods}
This thesis illiminate the online reprocessing case for thorium-fueled \gls{MSBR} through computational depletion analysis of 20-year-long reactor operational cycle. The current work implement a batch-wise approach where material is moved into or from the core at specific time intervals to modeling online reprocessing capabilities of \gls{MSBR}. To realize this method a continuous-energy Monte Carlo Burnup calculation code SERPENT 2 should be executed multiple times with specific depletion time step (e.g., 3 days) and depleted fuel composition after each depletion step  must be managed by external code to apply materials adds and removals. This approach was implemented in a SaltProc script \cite{andrei_rykhlevskii_arfc/saltproc:_2018} which has been developed by the thesis author and actively using for online reprocessing simulation with SERPENT 2 Monte Carlo code \cite{leppanen_serpent_2015}. 

To calculate whole-core depletion in the \gls{MSBR} the full-core high-fidelity model was developed using the continuous-energy SERPENT 2 Monte Carlo reactor physics software. It was employed to establish a batch-wise method for finding the equilibrium core composition and core depletion of the \gls{MSBR}. For this study equilibrium is defined as when effective muliplication factor of the full-core model with black boundary conditions and the $^{233}$U concentration in the fuel salt are both significantly invariant in time (i.e., vary less than percent over several months). The SaltProc script has capability to store number density of all isotopes in fuel salt which is helpful to estimate $^{232}$Th feed rate during \gls{MSBR} operation.

Obtained equilibrium fuel salt composition has been used to determine important operational and safety characteristics. Multiple steady-state computation was performed with full-core \gls{MSBR} model using the SERPENT 2 Monte Carlo code to compare neutron energy spectrum, power and breeding distribution, temperature coefficients of reactivity, control rod worths, six factors for both initial fuel loading and equlibrium fuel composition.
Another feature of the \gls{MSBR}, its circulating liquid fuel and corresponding delayed neutron precursor drift, is not treated here.

The structure of the thesis is as follows. In chapter 2, the history of \glspl{MSR} is recalled and their main features and associated consequences are listed. Particular focus is given to the \gls{ORNL} thorium-fueled thermal \gls{MSRE}, and the specifications of the two-fluid and the single-fluid \glspl{MSBR}. State-of-art literature review presented in the end of chapter 2. Chapter 3 covers the SERPENT 2 Monte Carlo overview and \gls{MSBR} model implemented for depletion calculations. Chapter 4 explains methodology of online reprocessing simulations and auxiliary code structure. In an attempt to avoid the pitfalls of a black box understanding and to identify method limitations at an early stage, governing equations and working principles are stated and discussed. Equilibrium seeking results and \gls{MSBR} important operational and safety parameters for both initial and equilibrium state are given in chapter 5. Additionally,
comparisons are made with available computational data from other works accompanied by brief analyses and discussions. Last chapter, summarizes the work and a conclusion is offered together with an outlook for future work on the topic. 
