\chapter[Introduction]{Introduction}

The thermal spectrum \gls{MSR} is an advanced type of reactor that consists of 
constantly circulating liquid fuel (i.e., mixture of LiF-BeF$_2$-ThF$_4$-UF$_4$ or 
LiF-BeF$_2$-ZrF$_4$-UF$_4$) and solid graphite moderator structures. This liquid 
fuel form leads to immediate advantages over traditional, 
solid-fueled, reactors. The molten-salt carrier with dissolved 
fissile and/or fertile material allows for online refueling and 
reprocessing. Thus, \glspl{MSR} can potentially operate years without shutdown, 
achieving maximum fuel utilization and outstanding neutron economy 
\cite{leblanc_molten_2010}. Moreover, this type of fuel does not need 
complex fabrication since the fissile material could be transported from any 
enrichment plant to the \gls{NPP} in the form 
of uranium hexafluoride (UF$_6$).  Additionally, \glspl{MSR} have a high level 
of inherent safety due to their characteristically 
the strong negative temperature coefficient of reactivity, near-atmospheric 
pressure in the primary loop, stable coolant, passive decay heat cooling, and 
small excess reactivity \cite{elsheikh_safety_2013}.

The thorium-fueled \gls{MSBR} was developed in the early 1970s by \gls{ORNL} 
specifically to realize the promise of the thorium fuel cycle which allows the 
use of natural thorium instead of enriched uranium as the fertile element. 
Thorium breeds the fissile $^{233}$U and avoids uranium enrichment 
\cite{robertson_conceptual_1971}. In the matter of nuclear fuel cycle, the 
thorium cycle produces a reduced quantity of plutonium and \glspl{MA} 
compared to the traditional uranium fuel cycle. Consequently, it may 
significantly increase proliferation resistance when the \gls{MSR} operates in the 
breeder regime. The \glspl{MSR} also could be employed as a converter reactor for 
transmutation of spent fuel from current \glspl{LWR}.

Recently, interest in \glspl{MSR} has resurged, with multiple new companies 
pursuing commercialization of \gls{MSR} designs\footnote{Examples include both 
liquid-fueled molten salt designs from Transatomic, Terrapower, Terrestrial, 
and Thorcon.}.
To further develop these \gls{MSR} concepts, particularly with respect to their  
strategies for online reprocessing and refueling, computational analysis methods capturing
their unique reactor physics and process chemistry are needed.
However, most contemporary nuclear reactor physics software is unable to 
perform depletion calculations in an online reprocessing 
regime. Powers \emph{et al.} suggested a novel method for conducting 
depletion simulations for \gls{MSR}. This suggested method takes into account 
fuel salt composition changes due to online 
reprocessing and refueling based on the deterministic computer code NEWT in 
SCALE \cite{powers_new_2013}. This approach was later used by Jeong \emph{et 
al.} to find an equilibrium fuel composition for the \gls{MSBR} and was 
validated with a \gls{MCNP}/CINDER90 model \cite{jeong_equilibrium_2016}. 

The current paper presents a single-cell model developed using the
continuous-energy Serpent 2 Monte Carlo reactor physics software. It was 
employed to establish a Serpent-based method for finding the equilibrium core composition and 
core depletion of the \gls{MSBR}.
All calculations presented in this paper were performed using the Serpent 2 
code version 2.1.29 with ENDF/B-VII.0
nuclear data \cite{leppanen_serpent_2012,chadwick_endf/b-vii.0:_2006}. Serpent 
2 is an improvement upon Serpent 1, and contains a complete 
redesign of memory management using hybrid OpenMP + MPI parallelization.  This 
hybrid parallelization is important in depletion calculations using computer 
clusters with multiple cores \cite{leppanen_serpent_2015}. This work used the 
built-in Serpent 2 depletion capabilities and its built-in online reprocessing 
subroutine. Another feature of the \gls{MSBR}, its circulating liquid fuel and 
corresponding delayed neutron precursor drift, is not treated here.
