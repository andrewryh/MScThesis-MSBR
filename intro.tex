\chapter[Introduction]{Introduction}

\section{Background and motivation}

Nowadays humankind has only few ways to generate reliable, nonintermittent base load power, namely, in no particular order: fossil fuel, hydroelectric, biomass, geothermal, and nuclear energy. Because of increasing global warming and climate change concerns, sources that have negligible CO$_2$ footprints represent crucial measures for control global temperature change. From environmental viewpoint these observations allow conclude that hydro and nuclear power are preferable ways to generate nonintermittent power\footnote{Biomass could be classified as renewable energy source but recent research has proved that it also produce considerable amount of CO$_2$ and not necessarily beneficial for the environment over time scales that are relevant when considering global climate change \cite{cherubini_co2_2011}.}. Nevertheless, potential for hydro power is strictly limited by local geographical conditions and have significant impact on environment, affecting land use, homes, and natural habitats in the dam area, consequently, one could argue that nuclear power should find wide usage around the world. 
 
In reality, nuclear power plants currenly generate 4.9\% of the global energy production \cite{noauthor_key_2017}, a figure which is projected to stay constant up to 2040 while electricity demand predicted to expand by 30\% \cite{noauthor_world_2017}. In contrast, contribution of coal in 2040 projected to be about 30\%. This visible fear towards large-scale switch from fossil to nuclear power based on a number factors such as concerns regarding safety, nuclear weapon prolifiration, radioactive waste treatment, and competitiveness with other sources of energy (i.e. renewables). Unfortunately, because of all these fears, a negative public attitude to nuclear has formed in many developed countries which makes challenging to advocate its zero emissions benefits.

\glspl{MSR} is one effort to overcoming the problems related to conventional \glspl{NPP}. As one of six advanced reactor concepts that have been chosen by the Generation IV International Forum (GIF) for further research and development, \glspl{MSR}  offering significant improvements ``in the four broad areas of sustainability, economics, safety and reliability, and proliferation resistance and physical protection" \cite{doe_technology_2002}. To achieve the goals formulated by the GIF, \glspl{MSR} attempt to simplify the reactor core and improve ingerent safety by using liquid coolant which is also a fuel\footnote{Molten-salt-cooled reactors with solid fuel are usually also referred to \glspl{MSR}. However, in this thesis, \glspl{MSR} are assumed to be reactors with liquid fuel which simultaneously serves as coolant.}. In the thermal spectrum \gls{MSR}, fluorides of fissile and/or fertile materials (i.e. UF$_4$, PuF$_3$ and/or ThF$_4$) are mixed with carrier salts to form a liquid fuel which is circulated in a loop-type primary circuit \cite{haubenreich_experience_1970}. This innovation leads to immediate advantages over traditional, solid-fueled, reactors. These include near-atmospheric pressure in the primary loop, relatively high coolant temperature, outstanding neutron economy, a high level of inherent safety, reduced fuel preprocessing, and the ability to continuously remove fission products and add fissile and/or fertile elements \cite{leblanc_molten_2010}. 

The thermal spectrum \gls{MSBR} was designed to realize the promise of the thorium fuel cycle, which uses of natural thorium instead of enriched uranium. Thorium breeds fissile $^{233}$U and avoids uranium enrichment. The mixture of LiF-BeF$_2$-ThF$_4$-UF$_4$-PuF$_3$ has a melting point of $499^\circ$C, a low vapor pressure at operating temperatures, and good flow and heat transfer properties \cite{robertson_conceptual_1971}. 

The \gls{MSBR} complex geometry is challenging to describe in software input and usually require major geometric 
approximations \cite{park_whole_2015}. 

\section{Methods}

\section{Objective and outline}

The thermal spectrum \gls{MSR} is an advanced type of reactor that consists of 
constantly circulating liquid fuel (i.e., mixture of LiF-BeF$_2$-ThF$_4$-UF$_4$ or 
LiF-BeF$_2$-ZrF$_4$-UF$_4$) and solid graphite moderator structures. This liquid 
fuel form leads to immediate advantages over traditional, 
solid-fueled, reactors. The molten-salt carrier with dissolved 
fissile and/or fertile material allows for online refueling and 
reprocessing. Thus, \glspl{MSR} can potentially operate years without shutdown, 
achieving maximum fuel utilization and outstanding neutron economy 
\cite{leblanc_molten_2010}. Moreover, this type of fuel does not need 
complex fabrication since the fissile material could be transported from any 
enrichment plant to the \gls{NPP} in the form 
of uranium hexafluoride (UF$_6$).  Additionally, \glspl{MSR} have a high level 
of inherent safety due to their characteristically 
the strong negative temperature coefficient of reactivity, near-atmospheric 
pressure in the primary loop, stable coolant, passive decay heat cooling, and 
small excess reactivity \cite{elsheikh_safety_2013}.

The thorium-fueled \gls{MSBR} was developed in the early 1970s by \gls{ORNL} 
specifically to realize the promise of the thorium fuel cycle which allows the 
use of natural thorium instead of enriched uranium as the fertile element. 
Thorium breeds the fissile $^{233}$U and avoids uranium enrichment 
\cite{robertson_conceptual_1971}. In the matter of nuclear fuel cycle, the 
thorium cycle produces a reduced quantity of plutonium and \glspl{MA} 
compared to the traditional uranium fuel cycle. Consequently, it may 
significantly increase proliferation resistance when the \gls{MSR} operates in the 
breeder regime. The \glspl{MSR} also could be employed as a converter reactor for 
transmutation of spent fuel from current \glspl{LWR}.

Recently, interest in \glspl{MSR} has resurged, with multiple new companies 
pursuing commercialization of \gls{MSR} designs\footnote{Examples include both 
liquid-fueled molten salt designs from Transatomic, Terrapower, Terrestrial, 
and Thorcon.}.
To further develop these \gls{MSR} concepts, particularly with respect to their  
strategies for online reprocessing and refueling, computational analysis methods capturing
their unique reactor physics and process chemistry are needed.
However, most contemporary nuclear reactor physics software is unable to 
perform depletion calculations in an online reprocessing 
regime. Powers \emph{et al.} suggested a novel method for conducting 
depletion simulations for \gls{MSR}. This suggested method takes into account 
fuel salt composition changes due to online 
reprocessing and refueling based on the deterministic computer code NEWT in 
SCALE \cite{powers_new_2013}. This approach was later used by Jeong \emph{et 
al.} to find an equilibrium fuel composition for the \gls{MSBR} and was 
validated with a \gls{MCNP}/CINDER90 model \cite{jeong_equilibrium_2016}. 

The current paper presents a single-cell model developed using the
continuous-energy Serpent 2 Monte Carlo reactor physics software. It was 
employed to establish a Serpent-based method for finding the equilibrium core composition and 
core depletion of the \gls{MSBR}.
All calculations presented in this paper were performed using the Serpent 2 
code version 2.1.29 with ENDF/B-VII.0
nuclear data \cite{leppanen_serpent_2012,chadwick_endf/b-vii.0:_2006}. Serpent 
2 is an improvement upon Serpent 1, and contains a complete 
redesign of memory management using hybrid OpenMP + MPI parallelization.  This 
hybrid parallelization is important in depletion calculations using computer 
clusters with multiple cores \cite{leppanen_serpent_2015}. This work used the 
built-in Serpent 2 depletion capabilities and its built-in online reprocessing 
subroutine. Another feature of the \gls{MSBR}, its circulating liquid fuel and 
corresponding delayed neutron precursor drift, is not treated here.
