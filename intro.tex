\chapter[Introduction]{Introduction}

\section{Background and motivation}

Nowadays humankind has only few ways to generate reliable, nonintermittent base load power: fossil fuel, hydroelectric, geothermal, and nuclear energy. Because of increasing global warming and climate change concerns, sources that have negligible CO$_2$ footprints represent crucial measures for control global temperature change. From an environmental viewpoint hydro and nuclear power are preferable ways to generate reliable power. Nevertheless, potential for hydro power is strictly limited by local geographical conditions and significantly impacts the environment, affecting land use, homes, and natural habitats in the dam area, hence, the only one option left is nuclear power. Nuclear power plants generate 4.9\% of global energy production \cite{noauthor_key_2017}, a figure which is projected to stay constant up to 2040 while electricity demand is predicted to expand by 30\% \cite{noauthor_world_2017}. Unfortunately, because of concerns regarding safety, nuclear weapon prolifiration, radioactive waste treatment, and competitiveness with other sources of energy (i.e. renewables), a negative public attitude to nuclear has formed in many developed countries which makes challenging to advocate its zero emissions benefits.

\glsentryfirstplural{MSR} are among the six advanced reactor concepts that have been chosen by the Generation IV International Forum (GIF) for further research and development. \glspl{MSR} offer significant improvements ``in the four broad areas of sustainability, economics, safety and reliability, and proliferation resistance and physical protection" \cite{doe_technology_2002}. To achieve the goals formulated by the GIF, \glspl{MSR} attempt to simplify the reactor core and improve inherent safety by using liquid coolant which is also a fuel\footnote{Molten-salt-cooled reactors with solid fuel can also be referred to \glspl{MSR}. However, in this thesis, \glspl{MSR} are assumed to be reactors with liquid fuel which simultaneously serves as coolant.}. In the thermal spectrum \gls{MSR}, fluorides of fissile and/or fertile materials (i.e. UF$_4$, PuF$_3$ and/or ThF$_4$) are mixed with carrier salts to form a liquid fuel which is circulated in a loop-type primary circuit \cite{haubenreich_experience_1970}. This innovation leads to immediate advantages over traditional, solid-fueled, reactors. These include near-atmospheric pressure in the primary loop, relatively high coolant temperature, outstanding neutron economy, a high level of inherent safety, reduced fuel preprocessing, and the ability to continuously remove fission products and add fissile and/or fertile elements \cite{leblanc_molten_2010}. 

The thermal spectrum \glsfirst{MSBR} was designed to realize the promise of the thorium fuel cycle, which uses natural thorium instead of enriched uranium. Thorium breeds fissile $^{233}$U and avoids uranium enrichment. The mixture of LiF-BeF$_2$-ThF$_4$-UF$_4$ has a melting point of $499^\circ$C, a low vapor pressure at operating temperatures, and good flow and heat transfer properties \cite{robertson_conceptual_1971}. 

The \gls{MSBR} complex geometry hard to describe in software input, and, usually, researchers make significant geometric simplifications to model it \cite{park_whole_2015}. Note that this thesis leverages extensive computational resources to avoid these geometric approximations and accurately capture breeding behavior.

\section{Objectives}
This thesis introduces the online reprocessing simulation code, SaltProc, which expands the capability of the continuous-energy Monte Carlo Burnup calculation code, SERPENT 2 \cite{leppanen_serpent_2015}, for simulation liquid-fueled \gls{MSR} operation \cite{andrei_rykhlevskii_arfc/saltproc:_2018}. The thesis also reports the application of the coupled SaltProc-SERPENT 2 system to the \gls{MSBR}, which represents the continuation of the work presented in \cite{rykhlevskii_full-core_2017, rykhlevskii_online_2017}. The main objective of the thesis herein is to analyse \gls{MSBR} neutronics and fuel cycle to find the equilibrium core composition and core depletion. The secondary objective is to compare predicted operational and safety parameters of the \gls{MSBR} at both the initial and equilibrium states. A tertiary goal is to demonstrate and prove that in a single-fluid two-region \gls{MSBR} conceptual design the undermoderated outer core zone II works as a virtual ``blanket", reduces neutron leakage and improves breeding ratio due to neutron energy spectral shift. Finally, $^{232}$Th feed rate will be determined and \gls{MSBR} fuel cycle performance will be analyzed.

\section{Methods}
This thesis establishes the online reprocessing case for the thorium-fueled \gls{MSBR} through computational depletion analysis of a 20-year-long reactor operational cycle. The current work implements a batch-wise approach where material is fed into or removed from the core at specific time intervals to model online reprocessing capabilities of \gls{MSBR}. In this work, SERPENT 2 was executed multiple times with specific depletion time step (e.g., 3 days). The depleted fuel composition after each depletion step  must be managed by external code to addand remove material. This approach was implemented in the SaltProc package \cite{andrei_rykhlevskii_arfc/saltproc:_2018} which has been developed by the thesis author for reprocessing simulation with the SERPENT 2 \cite{leppanen_serpent_2015}. 

To calculate whole-core depletion in the \gls{MSBR}, a full-core high-fidelity model was developed using SERPENT 2. The SaltProc batch-wise method was employed to find the equilibrium core composition and core depletion of the \gls{MSBR}. For this study equilibrium is defined as when the effective muliplication factor and the $^{233}$U concentration in the fuel salt are both significantly invariant in time (i.e., vary less than percent over several months). The SaltProc package stores the number density of all isotopes in the fuel salt in order to estimate $^{232}$Th feed rate during \gls{MSBR} operation.

The obtained equilibrium fuel salt composition was then used to determine important operational and safety characteristics. Multiple steady-state simulations were performed with the full-core \gls{MSBR} SERPENT 2 model. Neutron energy spectrum, power and breeding distribution, temperature coefficients of reactivity, control rod worths, six factors were then compared for both the initial and equlibrium fuel compositions.
Another feature of the \gls{MSBR}, delayed neutron precursor drift corresponding to its circulating liquid fuel, is not treated here.

The structure of the thesis is as follows. In chapter 2, the history of \glspl{MSR} is recalled and their main features and associated consequences are listed. Particular focus is given to the \gls{ORNL} thorium-fueled thermal \gls{MSRE}, and the specifications of the two-fluid and the single-fluid \glspl{MSBR}. A review of the state of the art in modeling and simulating liquid fuelled \glspl{MSR} such as the \gls{MSBR} is presented in the end of chapter 2. Chapter 3 covers the SERPENT 2 Monte Carlo software and the \gls{MSBR} model implemented for depletion calculations. Chapter 4 explains the online reprocessing simulation methodology and auxiliary code structure. In an attempt to avoid the pitfalls of a black box understanding and to identify method limitations at an early stage, governing equations and working principles are stated and discussed. Equilibrium-seeking results as well as important operational and safety \gls{MSBR} parameters for both the initial and equilibrium states are given in chapter 5. Additionally,
comparisons are made with available computational data from other works accompanied by brief analyses and discussions. The last chapter summarizes the contribution of this thesis and a conclusion is offered together with an outlook for future work on the topic. 
