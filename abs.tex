Current interest in advanced nuclear energy and \gls{MSR} concept has enhanced demand in building the tools to analyse these systems. A Python script SaltProc has been developed to simulate \gls{MSR} online reprocessing by modeling the changing isotopic composition of an irradiated fuel salt using Monte Carlo code SERPENT 2 for neutron transport and depletion calculations. SaltProc capabilities include a generic geometry capable of modeling multi-region and multi-flow reactors, time-dependent feeds and removals, and specific separation efficiency for any flow. Generally applicable  capabilities are illuminated in this thesis in three applied problems: (1) simulating the startup of a thorium-fueled \gls{MSBR} fuel cycle to find equilibrium fuel composition (when muliplication factor of the full-core model and the $^{233}$U concentration in the fuel salt are both significantly invariant in time); (2) determining the effect of the online reprocessing on \gls{MSBR} operations; and (3) determining \gls{MSBR} fuel cycle performance by computing $^{232}$Th feed rate over 20 years of operation and comparing with available data.

In the first application, full-core depletion in the \gls{MSBR} demonstrated that (1) equilibrium fuel composition could be achieved after 16 years of operation and (2) multiplication factor is invariant in time after 6 years of operation. In the second application, fuel salt irradiation even with simulated fission products removal and fissile/fertile feed causes considerable neutron energy spectrum hardening; this spectral shift has negative impact on safety parameters (temperature reactivity feedback, reactivity control system worth). In the third application, the average $^{232}$Th feed rate throughout 20 years of operation is 2.39 kg/day or 100 g/GWh$_e$ which is a good agreement with most recent research. Negative effects of neutron energy spectrum hardening during \gls{MSBR} operation should be taken into account for neutronics, multi-physics and fuel cycle performance analysis.