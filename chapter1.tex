\chapter[Molten Salt Reactors]{Molten Salt Reactors}


\section{History}

Developing of \glspl{MSR} started in the late 1940's as part of the United States' program to design a nuclear powered airplane. Particularly  liquid fuel appeared to offer number of advantages, and experiments to demonstrate the feasibility of molten salt fuels were begun in 1947 on ``the initiative of V.P.Calkins, Kermit Anderson, and E.S.Bettis. At the enthusiastic urging of Bettis and on the recommendation of W.R.Grimes, R.C.Briant adopted molten fluoride salts in 1950 as the main line effort of the \glsfirst{ORNL}'s Aircraft Nuclear Propulsion program." The flourides appeared exceptionally suitable because they have high solubility for uranium, are among the most stable of chemical compounds, have low vapor pressure even at temperature more than 1300$^{\circ}$C, have fairly good hydraulic and thermal properties, do not react furiously with air of water, are not damaged by hight neutron fluxes, and are inert to some common structural materials \cite{rosenthal_molten-salt_1970}.

A small test reactor, the \gls{ARE}, was built at Oak Ridge site to probe the use of molten flouride fuels for aircraft propulsion reactors and to study the nuclear stability of the circulating fuel system. Fuel salt for the \gls{ARE} was a mixture of NaF, ZrF$_4$, and UF$_4$, BeO served as moderator, and all the piping was nickel-chromium alloy Inconel. The experiment was successful: in 1954 the \gls{ARE} was operated for 9 days at steady-state outlet temperatures up to 860$^{\circ}$C and at powers up to 2.5 MW$_{(th)}$. No mechanical or chemical problems were observed, and the reactor was found to be stable and self-regulating.

Great potential of \glspl{MSR} for civilian power application was recognized from the beginning of Aircraft Nuclear Propulsion program, and in 1956 H.G.MacPherson founded a group to study the technical characteristics, nuclear performance, and economics of molten salt converting and breeding reactors. After few years of research with number of concepts, MacPherson and his colleagues concluded that graphite-moderated thermal reactors operating on a thorium fuel cycle would be the best choice for applying molten salt systems for producing economic energy. Breeding $^{233}$U from $^{232}$Th was found to give better performance in a molten salt thermal reactor neutron energy spectrum than a uranium fuel cycle in which depleted uranium ($^{238}$U) is the fertile material and fissile $^{239}$Pu is produced and recycled. Homogeneous reactor designs that have entire core consist of liquid sal were rejected because the limited moderation by the salt components did not prove to make good thermal reactor comparing with one moderated by graphite. Furthermore, intermediate spectrum reactors did not appear to have high enough breeding ratios to compensate for their higher inventory of fuel. Later studies of fast spectrum molten salt reactors \cite{kasten_mosel_1964} has shown that effective breeding could be obtained with extremely high power densities that needed to avoid excessive fissile inventories. Acceptable  power densities appeared challenging to achieve without using novel and untested heat transfer technologies.

Two types of graphite-moderated reactors were selected by MacPherson's group for further research: single-fluid reactors in which thorium and uranium are dissolved in the same carrier salt, and two-fluid design in which a fertile salt accommodated $^{232}$Th is separated from the fissile salt which contains $^{233}$U and/or $^{239}$Pu as initial fissile load for reactor startup. The two-fluid reactor could operates as breeder but construction materials separating flows would significantly deteriorate neutron economy and, consequently, breeding ratio. The single-fluid design is much simplier, easier to build and offers lower power costs, even for that time technology which could only achive breeding ratio slightly below 1.0. The chemical reprocessing method namely fluoride volatility process, which separates uranium from flouride salts, had been already demonstrated during \gls{ARE} for recovery uranium from \gls{ARE} fuel salt and might be used for partial reprocessing of salts from another type of reactor.

U.S. Atomic Energy Commission task force have considered results of the \gls{ORNL} research and made a comparative evaluation of fluid-fueled reactors early in 1959. One conclusion of the task force was that the \gls{MSR} even limited in potential breeding gain, had ``the highest probability of achieving technical feasibility." \cite{noauthor_report_1959}

The concept of liquid-fueled \gls{MSR} was initially proposed by \gls{ORNL} in the projects of the \gls{ARE} 

Molten salt reactor concepts garnered considerable interest in the 1950s and 60s
with development of the \gls{ARE} and later the \gls{MSRE} at \gls{ORNL}.  With
the inclusion of the \gls{MSR} among the Generation-IV reactor designs
\cite{gif_generation_2008,gif_generation_2015}, this reactor concept has gained renewed research
interest in the past decade, with many new nuclear companies proposing both
fluid-fuelled and solid-fuelled commercial \gls{MSR} designs
\cite{hyde_liquid_2015,leblanc_integral_2015,thorcon_-able_2017,scarlat_design_2014,transatomic_power_corporation_neutronics_2016}.
The key advantages of \gls{MSR}s generally pertain to improved fuel utilization
and reactor safety. In contrast to legacy reactors, only moderator fast neutron
damage and fuel chemistry evolution limit burnup. A clever configuration of moderator
as in \cite{engel_conceptual_1980} can enable reactor operation without opening the vessel
for thirty or more years. Further, several fission products selectively precipitate onto
nickel surfaces in fluoride salt, as documented in \cite{engel_conceptual_1980}, thus reducing
unwanted neutron absorption. Lastly, the epithermal spectrum of graphite-moderated salt reactors
incinerates plutonium more efficiently, thus reducing long-lived transuranic waste production \cite{engel_conceptual_1980}.
The sum of these characteristics implies the \gls{MSR} uniqely ameliorates spent fuel burden whilst
extending nuclear fuel resources. To top all these benefits off, xenon transients become moot in
\gls{MSR}s due to its insolubility in salt, thus narrowing transient analysis focus to thermalhydraulic
concerns.

Simulation tools developed by many authors successfully describe steady-state and
transient behavior of myriad \gls{MSR} concepts. Krepel et al. extended the in-house \gls{LWR}
diffusion code DYN3D to consider drift of delayed neutron precursors alongside
the reactor temperature profile, re-casting the extended code as
DYN3D-MSR \cite{krepel_dyn3d-msr_2007}. That work compared DYN3D-MSR against
experimental \gls{MSRE} data and then used it to simulate local fuel channel
blockages as well as local temperature perturbations.
\section{The thorium fuel cycle}

\section{Literature review}

