\chapter[Conclusions and furture work]{Conclusions and future work}
This thesis introduces the open source \gls{MSR} simulation code SaltProc. The SaltProc modeling and simulation tool expands the capability of a continuous-energy Monte Carlo Burnup calculation code SERPENT 2 for simulating liquid-fueled \gls{MSR} operation \cite{andrei_rykhlevskii_arfc/saltproc:_2018}. Benefits of SaltProc include generic geometry modeling, multi-flow capabilities, time-dependent feed and removal rates, and ability to specify removal efficiency. The main goal has been to assess the ability of this tool to find equilibrium fuel salt composition (e.g., number density of major isotopes vary less than 1\% over several years). A secondary goal has been to compare predicted operational and safety parameters (e.g., neutron energy spectrum, power and breeding distribution, temperature coefficients of reactivity) of the \gls{MSBR} at initial and found equilibrium state. A tertiary study goal has been to demonstrate and prove that in one-fluid two-region \gls{MSBR} conceptual design the outer core zone II serves as breeding ``blanket" to catch leaked from the central core zone I neutrons and improve breeding ratio.

In order to assemble the SaltProc tool, a full-core high-fidelity benchmark model of the \gls{MSBR} was implemented in the SERPENT 2. The purpose of the full-core fidelity model instead of simplified single-cell model \cite{rykhlevskii_online_2017}, \cite{betzler_molten_2017} was to precisely describe two-region \gls{MSBR} concept design that allowed accurately represent breeding in a ``blanket" (outer core zone). When running depletion calculation, the most important fission products and $^{233}$Pa are removed and fertile/fissile materials are added to fuel salt every 3 days while rare earths, volatile flourides and seminoble metals removal interval was more than month. 

The results of this study indicate that from the depletion calculation the effective multiplication factor slowly decreases from 1.075 and reaches to the equilibrium state with the factor about 1.02 after approximately 6 years of operation. At the same time, concentration of $^{233}$U, $^{232}$Th, $^{233}$Pa, $^{232}$Pa composition changes insignificantly after approximately 2500 days of operation. Particullarly, $^{233}$U number density fluctuates less than 0.8\% from 16 to 20 years of operation, consequently, it might be assumed that the core reaches to the quasi-equlibrium state after 16 years of the fuel irradiation. On the other hand, wide diversity of nuclides, including fissile isotopes (e.g. $^{233}$U, $^{239}$Pu) and non-fissile strong absorbers (e.g. $^{234}$U), keep accumulating in the core. The results in this thesis shows that trully equilibrium materials composition cannot exist but the balance between negative effects of strong absorbers accumulation and new fissile materials production might be achieved to keep the reactor critical.

The next most obvious finding to emerge from the analysis of initial and equilibrium materials composition is that neutron energy spectrum is harder for equilibrium state because significant amount of heavy fission products were accumulated in the \gls{MSBR} core. Moreover, neutron energy spectrum in the central core region is much softer than in outer core region due to lower moderator-to-fuel ratio in outer zone, and this distribution not changes during reactor operation. Finally, the epithermal and thermal spectrum needed to effectively breed $^{233}$U from $^{232}$Th because radiative capture cross section of thorium-232 monotonically decreases from $10^{-10}$ MeV to $10^{-5}$ MeV. Harder spectrum in the outer core region tends to significantly increase resonance absorption in thorium and decrease the absorptions in fissile and construction materials. 

The calculated heat power spatial distribution of the \gls{MSBR} shows that 98\% of the fission power is generated in central zone I, and neutron energy spectral shift did not cause any notable changes in power distribution. The neutron capture reacton rate spatial distribution for fertile $^{232}$Th, which quantified breeding in the core, confirms that the most of breeding occurs in an outer, undermoderated, region of the \gls{MSBR} core. Moreover, calculated thorium-232 feed rate gradually decreases from about 2.7 kg/day at the beginning of operation to 2.4 kg/day at the end of 20-years timeframe. Finally, the average $^{232}$Th refill rate throughout 20 years of operation is approximately 2.39 kg/day or 100 g/GWh$_e$ in this study which is a good agreement with most recent online reprocessing analysis by \gls{ORNL} \cite{betzler_molten_2017}.

Comparisons of the safety parameters were made for initial fuel loading and equilibrium materials composition with SERPENT 2 code. It is noted that neutron energy spectrum hardening over the fuel depletion and this spectral shift couses changes in the reactor behaviour. The total temperature coefficient is relatively large and negative at the initial and equilibrium state but the magnitude decreases throughout reactor operation from $-3.10$ to $-0.94$ pcm/K, and moderator temperature coefficient is positive and also decreases during fuel depletion. From reactivity control system efficiency results, the safety rod integral worth decreases by approximately 16.2\% over 20 years of operation, while graphite rod integral worth staying the same. Summing up, neutron energy spectrum hardening during fuel salt depletion has considerable negative impact on \gls{MSBR} stability and controllability, and should be taken into consideration in further analysis of accident transient scenarios.

Continued research into SaltProc-SERPENT 2 and related topics could progress in a number of different directions. First and foremost efforts should be made at reprocessing parameters (e.g. time step, feeding rate, protactinium removal
rate) optimization to achieve the best fuel utilization, breeding ratio or safety characteristics. This might be performed using additional script which would change input parameter by small increment, run workbench and analyse output to determine optimal configuration. Furthermore, existing optimisation framework RAVEN might be employed for this optimisation study \cite{alfonsi_raven_2013}.

Only the semi-batch online reprocessing approach has been treated in this thesis. However, SERPENT 2 Monte Carlo code recently was extended for trully continuous online fuel reprocessing simulation \cite{aufiero_extended_2013}. This extension must be verified against existing SaltProc/SERPENT or ChemTriton/SCALE workbench, and could be employed to immediate removal of fission product gases (e.g., Xe, Kr) which has strong negative impact on core lifetime and breeding efficiency. Finally, using build-in SERPENT 2 Monte Carlo code online reprocessing \& refueling material burnup routine would significantly speedup computer-intensive full-core depletion simulations.

Lastly, an additional area to explore is the accident safety analysis which requires to develop multi-physics model of \gls{MSBR} in the coupled neutronics/ thermal-hydraulics code Moltres \cite{lindsay_introduction_2018}. Existing full-core SERPENT 2 model and equilibrium fuel material composition would be employed to problem-oriented nuclear data libraries generation for further usage in accident transient analysis. The final goal of this effort is develop fast-running computational model which would aim at studying the dynamic behavior of generic \gls{MSR}, performing detailed safety analysis and design optimization.

