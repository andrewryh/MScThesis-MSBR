\chapter[Steady state full-core MSBR criticality simulation]{Steady state full-core MSBR criticality simulation}

\section{Serpent 2 code overview}

\section{Molten Salt Breeder Reactor description}

\subsection{Core Zone I}

The central portion, called Zone I, is made up of 1320 graphite elements, each 
$10.16$cm$\times$10.16cm$\times$396.24cm.
In Zone I, 13\% of the volume is fuel salt and 87\% is graphite. Zone I is 
composed of 1320 graphite cells and 4 channels for control rods: two for 
graphite rods which both regulate and shim during normal operation, and two
for backup safety rods to assure sufficient negative reactivity for emergency 
situations.

These graphite elements have a mostly rectangular shape with lengthwise ridges 
at each corner that leave space for salt flow elements. Various element sizes 
reduce the peak damage flux and power density in the center of the core prevent 
local graphite damage. Figure~\ref{fig:zone12A} demonstrates the elevation and 
sectional views of graphite elements exactly as they are represented in this 
Monte Carlo model.

\subsection{Core Zone II}
The undermoderated zone, Zone II, surrounds Zone I.
Combined with the bounding radial reflector, Zone II serves to diminish neutron 
leakage. This zone is formed of two kinds of elements: elements like those in 
Zone I with a larger channel diameter (Zone II-A), and radial
graphite slats (Zone II-B). 

Zone II is Zone II 37\% salt by volume and each element has a fuel channel 
diameter of 6.604cm. It is divided into two different zones: Zone II-A and Zone 
II-B. The graphite elements for Zone II-A are prismatic. Zone II-B elements are 
rectangular slats spaced far enough apart to provide the 0.37 fuel salt volume 
fraction. Fig.~\ref{fig:zone2B} additionally shows the 5.08cm-wide annular 
space between the core graphite and the radial reflector graphite. The annulus 
contains 100\% fuel salt and serves to reduce the damage flux at the internal 
surface of the graphite reflector blocks. The reactor Zone II-B graphite 
5.08cm-thick slats vary in the radial dimension (average width is 26.67cm) but 
are reconstructed without any approximation. From the ORNL report 
\cite{robertson_conceptual_1971}, the suggested design of Zone II-B has 8 
irregularly-shaped graphite elements every 45$^\circ$ as well as salt channels. 
These graphite elements were simplified into right-circular cylindrical shapes 
with central channels. This is the only simplification made to the \gls{MSBR} 
conceptual geometry in this work.

\subsection{Material composition}
The fuel salt, the reactor graphite, and the modified HastelloyN are materials 
unique of the \gls{MSBR} and were created at \gls{ORNL}. The initial fuel salt 
loading composition is LiF-BeF$_2$-ThF$_4$-$^{233}$UF$_4$ (71.8-16-12-0.2 mole 
\%). The lithium in the molten salt fuel is a fully enriched $^{7}$Li because 
$^{6}$Li is a very strong neutron poison and becomes tritium upon neutron 
capture. For cross section generation, ENDF/B-VII was employed 
\cite{chadwick_endf/b-vii.0:_2006}. The specific temperature was fixed for each 
material to correctly model the Doppler-broadening of resonance peaks when 
Serpent generate problem-oriented nuclear data library.
The isotope composition of each material at the initial state was described in 
detail in the MSBR conceptual design study \cite{robertson_conceptual_1971} 
and has been applied to Serpent model without any modification. %

\section{Results}